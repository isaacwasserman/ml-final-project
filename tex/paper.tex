\documentclass{article}
\usepackage[preprint]{neurips_2022}
\usepackage[utf8]{inputenc} % allow utf-8 input
\usepackage[T1]{fontenc}    % use 8-bit T1 fonts
\usepackage{hyperref}       % hyperlinks
\usepackage{url}            % simple URL typesetting
\usepackage{booktabs}       % professional-quality tables
\usepackage{amsfonts}       % blackboard math symbols
\usepackage{nicefrac}       % compact symbols for 1/2, etc.
\usepackage{microtype}      % microtypography
\usepackage{xcolor}         % colors
\bibliographystyle{plain}

\title{Modeling the Graphotactics of Low-Resource Languages Using Sequential GANs}

\author{%
  Isaac Wasserman\\
  Department of Computer Science\\
  Haverford College\\
  Haverford, PA 19041\\
  \texttt{iwasserman@haverford.edu} \\
}

\begin{document}

  \maketitle
  \begin{abstract}
    Generative Adversarial Networks (GANs) have been shown to aid in the creation of artificial data in situations where large amounts of real data are difficult to come by. This issue is especially salient in the computational linguistics space, where researchers are often tasked with modeling the complex morphologic and grammatical processes of low-resource languages. This paper will discuss the implementation and testing of a GAN that attempts to model and reproduce the graphotactics of a language using only 100 example strings. These artificial, yet graphotactically compliant, strings are meant to aid in modeling the morphological inflection of low-resource languages.
  \end{abstract}
  \section{Introduction}
  \subsection{Task}
  In 2019, Anastasopoulos and Neubig made waves with their multilingual morphological inflection model for low resource languages \cite{CMU} that they submitted to the SIGMORPHON 2019 shared task \cite{sigmorphon2019}. All models submitted were pretrained on high resource languages of similar ancestry to the target language, allowing many models to greatly exceed the performance of previous attempts at low-resource morphological inflection. However, what allowed Anastasopoulos and Neubig's model to outperform other submissions was its use of data ``hallucination''.
  
  To perform this hallucination, they aligned the lemma with its inflected form, extracted the stem, and generated new artificial examples by replacing this stem with randomly generated strings (in the language's alphabet) of equal length.\footnote{The alignment process assumes that the lemma and inflected form share a common substring.} Though a seemingly haphazard method of data augmentation, this approach allowed for an additional 10\% accuracy, on average, when tested against versions of the model that only used cross-lingual transfer.

  Surely, a more well informed approach to stem generation would further improve the accuracy of the inflectional model. Given the demonstrated ability of GANs to produce photorealistic, yet completely contrived images, they are potentially ideal for such a task.

  \subsection{Generative Adversarial Networks}
  Generative adversarial networks are a class of unsupervised machine learning architectures, most commonly used for image generation. These networks consist of a generator and a discriminator that are trained simultaneously on a set of data representing a class or domain; this domain could be anything from photos of human faces to time series of hourly temperatures.\footnote{Technically speaking, the generator and discriminator are most often trained one after another on a repeated basis.} The generator is tasked with producing ``fake'' examples that are within this domain without ever seeing any real examples from the training set. Meanwhile, the discriminator is fed a combination of fake examples (from the generator) and real examples and is tasked with classifying them as real or fake. The respective goals of the generator and discriminator constitute a zero-sum game, in which the generator is constantly trying to outsmart the discriminator, while the discriminator hones its ability to distinguish between in-domain and out-of-domain examples.

  Though GANs are most often applied to image data (as in the popular CycleGAN \cite{cyclegan}, StyleGAN \cite{stylegan}, and DiscoGAN \cite{discogan}), the same logic is also applicable to other types of data. For example, in 2017, Esteban et al. developed a pair of recurrent GANs which they applied to medical time series data \cite{rcgan}, and in 2016, Yu et al. developed a GAN architecture made specifically for sequences and language generation, utilizing techniques from reinforcement learning \cite{seqgan}.

  \subsection{GANs for Data Augmentation}
  Although GANs are most popularly used for domain transfer, the same basic architecture is also a good candidate for a data augmentation strategy called hallucination in which new (fake) training examples based on a small number of existing (real) examples.

  Though theoretically, the addition of these fake examples should not improve the performance of a discriminative model (since they can be no more representative of the true distribution than the examples they are based on), an empirical study by Antoniou et al. observed accuracy improvements on multi-class classification of up to 13\% on benchmark low-resource datasets such as Omniglot \cite{antoniou}. These performance gains are supported by a fairly extensive body of similar evidence-based studies on GAN-based data augmentation in low-resource settings. These studies are, more often than not, concerned with medical imaging, in which segmentation is a more salient issue than classification.

  Shin et al., 2018 \cite{shin} leveraged the popular Pix2Pix \cite{pix2pix} architecture to augment a brain-tumor segmentation dataset. This task was considerably more complex as it involved the hallucination of image pairs. However, training their segmentation model on a combination of real and fake data, they observed performance improvements of up to 16\% over unaugmented data (minimum improvement of ~1\%). They also tested the effects of an entirely synthetic dataset; however, this resulted in a significant loss of performance compared to the baseline. Despite the improvements realized when compared to their baseline segmentor, even their best model was unable to outperform the best-in-class reference model \cite{wang}.

  Sandfort et al., 2019 \cite{sandfort} carried out a similar study using the also popular CycleGAN architecture \cite{cyclegan} to segment anomolous CT scans of kidneys, livers, and spleens. On average, this study showed no appreciable difference between the performance gains afforded by traditional augmentation and GAN-based augmentation. However, when the resulting models were tested on images that were out-of-distribution, they found that while the baseline model scored 0.101, the traditionally augmented and GAN augmented models received scores of 0.535 and 0.747 respectively.\footnote{While the training set was based on CTs with contrast, these out-of-distribution images came from scans performed without contrast.} This increased flexibility is interesting as it suggests that the augmented examples constitute a wider domain than the examples they are based on. From a theoretical standpoint, this would have to be possible if GAN augmented datasets were to outperform others.

  While all of the other related works cited have used GANs to create synthetic image data, Gupta, 2019 \cite{gupta} applies GAN-based augmentation to language data for the purpose of sentiment analysis. Unlike the other applications, labeled language data for sentiment analysis is not scarce. The datasets used each contain between 2000 and 4000 real examples, and the classifier was pre-trained on a dataset of over 1.6 million examples. Real and fake examples were combined in a somewhat nontraditional way; instead of concatenating the real and synthetic datasets, separate classifiers were trained for each and their outputs were combined via bagging. In these experiments, accuracy was only improved by up to 1.2\% when fake examples were added. Though these gains are less impressive than others cited above, these results support the idea that GAN-based augmentation can be applied to domains outside of image data.
  
  \section{Method}
  

  \bibliography{sources}

\end{document}